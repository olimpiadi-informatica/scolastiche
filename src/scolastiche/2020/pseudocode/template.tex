% Template per le Selezioni Scolastiche delle OII

\documentclass[a4paper]{article}
\usepackage[utf8x]{inputenc}
\usepackage{lmodern}
\usepackage[margin=0.8in]{geometry} % il minimo sensato è 0.25in vedi qua: https://stackoverflow.com/a/3513476/747654
\usepackage[italian]{babel}
\usepackage{graphicx} % per \includegraphics{}
\usepackage{tabularx} % per tabularx, una versione di tabular che permette di "espandere" la colonna
\usepackage{textcomp} % per il simbolo \textdegree{}
\usepackage[hidelinks]{hyperref}
\usepackage{enumitem}
\usepackage{etoolbox}
\usepackage{xcolor}
\usepackage{soul}
\usepackage{amsmath}
\usepackage{float}
\usepackage[section]{algorithm}
\usepackage{algorithmicx}
\usepackage{algpseudocode}
\usepackage{py2tex}

% \noindent globale
\setlength\parindent{0pt}
\setlength\parskip{2pt}

% \myul{}: underline + colore blu
\newcommand{\myul}[2][black]{\setulcolor{#1}\ul{\textbf{#2}}\setulcolor{black}}

% \cbox{}: un quadratino per scriverci dentro un carattere o per farci sopra una X
\newcommand\cbox{\fbox{\rule{0.1in}{0pt}\rule[-0.2ex]{0pt}{2ex}}}

% nuovo tipo di colonna per tabularx che "espande" come 'X' ma centra come 'c'
\newcolumntype{Y}{>{\centering\arraybackslash}X}

% aggiusta leggermente l'altezza delle righe in tabularx (necessario per le \cbox{})
\setlength{\extrarowheight}{1pt}
\renewcommand{\arraystretch}{1.1}
\renewcommand{\tabularxcolumn}[1]{m{#1}}

% nasconde i numeri di pagina
\pagestyle{empty}

% ambente all'interno del quale non è possible andare a pagina nuova
\newenvironment{absolutelynopagebreak}
{\par\nobreak\vfil\penalty0\vfilneg
\vtop\bgroup}
{\par\xdef\tpd{\the\prevdepth}\egroup
\prevdepth=\tpd}

% definisce l'ambiente "esercizio"
\newcounter{indicedellesercizio}[section]
\newenvironment{esercizio}[1]{}{}

\newcommand{\rispostaaperta}{}
\newcommand{\rispostachiusa}[4]{
\begin{absolutelynopagebreak}
    \begin{enumerate}
        \item #1
        \item #2
        \item #3
        \item #4
    \end{enumerate}
    \emph{Inserisci nel form il numero della risposta (1, 2, 3 oppure 4).}
\end{absolutelynopagebreak}
}
\newcommand{\sezione}[1]{
\clearpage
\begin{center}
    \Huge #1
\end{center}
}
\newcommand{\sezionelogicomatematica}{\sezione{Esercizi di carattere logico matematico}}
\newcommand{\sezioneprogrammazione}{\sezione{Esercizi di programmazione}}
\newcommand{\sezionealgoritmi}{\sezione{Esercizi di carattere algoritmico}}

\begin{document}

\begin{algorithmic}[1]
\State{\PyAnnotation{\PyName{i}}{integer}}
\State{\PyAnnotation{\PyName{j}}{integer}}
\State{\PyAnnotation{\PyName{k}}{integer}}
\State{\PyAnnotation{\PyName{l}}{integer}}
\State{\PyAnnotation{\PyName{s1}}{integer}}
\State{\PyAnnotation{\PyName{s2}}{integer}}
\State{\PyAnnotation{\PyName{x}}{integer}}
\State{\PyAssign{\PyName{x}}{-\PyExpr{\PyName{Infinity}}}}
\State{\PyAssign{\PyName{i}}{\PyExpr{\PyNum{0}}}}
\While{\PyExpr{\PyName{i} \PyLt \PyName{n} \PySub \PyNum{1}}}
  \State{\PyExpr{\PyCall{print}{\PyName{i}}}}
  \State{\PyAssign{\PyName{j}}{\PyExpr{\PyName{i} \PyAdd \PyNum{1}}}}
  \While{\PyExpr{\PyName{j} \PyLt \PyName{n}}}
    \State{\PyExpr{\PyCall{print}{\PyName{j}}}}
    \State{\PyAssign{\PyName{k}}{\PyExpr{\PyName{i} \PyAdd \PyNum{1}}}}
    \State{\PyExpr{\PyCall{print}{\PyName{k}}}}
    \While{\PyExpr{\PyName{k} \PyLt \PyName{j}}}
      \State{\PyAssign{\PyName{s1}}{\PyExpr{\PyNum{0}}}}
      \State{\PyAssign{\PyName{s2}}{\PyExpr{\PyNum{0}}}}
      \State{\PyAssign{\PyName{l}}{\PyExpr{\PyName{i}}}}
      \While{\PyExpr{\PyName{l} \PyLt \PyName{k}}}
        \State{\PyAssign{\PyName{s1}}{\PyExpr{\PyName{s1} \PyAdd \PySubscript{\PyName{v}}{\PyName{l}}}}}
        \State{\PyAssign{\PyName{l}}{\PyExpr{\PyName{l} \PyAdd \PyNum{1}}}}
      \EndWhile%
      \While{\PyExpr{\PyName{l} \PyLtE \PyName{j}}}
        \State{\PyAssign{\PyName{s2}}{\PyExpr{\PyName{s2} \PyAdd \PySubscript{\PyName{v}}{\PyName{l}}}}}
        \State{\PyAssign{\PyName{l}}{\PyExpr{\PyName{l} \PyAdd \PyNum{1}}}}
      \EndWhile%
      \If{\PyExpr{\PyName{s1} \PySub \PyName{s2} \PyGt \PyName{x}}}
        \State{\PyAssign{\PyName{x}}{\PyExpr{\PyName{s1} \PySub \PyName{s2}}}}
      \EndIf%
      \State{\PyAssign{\PyName{k}}{\PyExpr{\PyName{k} \PyAdd \PyNum{1}}}}
    \EndWhile%
    \State{\PyAssign{\PyName{j}}{\PyExpr{\PyName{j} \PyAdd \PyNum{1}}}}
  \EndWhile%
  \State{\PyAssign{\PyName{i}}{\PyExpr{\PyName{i} \PyAdd \PyNum{1}}}}
\EndWhile%
\State{\PyExpr{\PyCall{print}{\PyName{x}}}}
\end{algorithmic}


\end{document}